\documentclass[a4paper,12pt]{article}
\usepackage{amsmath}
\usepackage{amssymb}
\usepackage{amsthm}
\usepackage[makeroom]{cancel}
\usepackage{hyperref}

\begin{document}
\title{Estimating $e$ on a checkerboard}
\author{Samuel Hunter}
\date{8 September 2019}
\maketitle

\section*{Method}
Adapted from \url{https://twitter.com/JohnAllenPaulos/status/743085219511554049}
Take an $n$ x $n$ checkerboard and label its squares 1 to $m=n^2$. Randomly generate $m$ whole numbers between 1 and $m$. For each of the numbers generated, place a penny on the square with the same number. The ratio of total squares, $m$, to squares with no pennies, $U$, is approximately $e$.

\section*{Solution}
For the $i$th square ($1 \leq i \leq m$), let $X_i$ be a random variable denoting the number of pennies on the $i$th square after all pennies have been placed.

\begin{equation}
	X_i \sim Binomial(m, \frac{1}{m})
\end{equation}

\noindent Define a new indicator random variable
\begin{equation}
	Y_i=
	\begin{cases}
		0, & \text{if}\ X_i=0 \qquad \text{square is empty} \\
		1, & \text{otherwise}
	\end{cases}
\end{equation}

Then $Y_i$ is a Bernoulli trial with expected value
\begin{equation}
	\mathbb{E}[Y_i] = 1*\mathbb{P}(X_i=0) + 0*\mathbb{P}(X_i>0) = (1-\frac{1}{m})^m
\end{equation}

\noindent Now over all of the squares, the number of empty squares is given by
\begin{equation}
	U = \sum_{i=1}^{m}{Y_i}
\end{equation}
\indent Take its expected value:
\begin{align}
	\mathbb{E}[U] &= \mathbb{E}[\sum_{i=1}^{m}{Y_i}]\nonumber\\
	&=\sum_{i=1}^{m}{\mathbb{E}[Y_i]} &&\text{by linearity of expectation}\nonumber\\
	&=\sum_{i=1}^{m}{(1-\frac{1}{m})^m} &&\text{by (3)}\nonumber\\
	&=m(1-\frac{1}{m})^m
\end{align}

\noindent Remember, we are interested in the ratio
\begin{equation*}
	\frac{\text{total squares}}{\text{empty squares}} = \frac{m}{U}
\end{equation*}
\indent which on average is
\begin{equation*}
	\frac{m}{\mathbb{E}[U]} = \frac{\cancel{m}}{\cancel{m}(1-\frac{1}{m})^m} = (1-\frac{1}{m})^{-m}
\end{equation*}
\indent Taking the limit as total squares increases without bound,
\begin{equation}
	\lim_{m\to\infty} (1-\frac{1}{m})^{-m} = (\lim_{m\to\infty} (1-\frac{1}{m})^m)^{-1} = (e^{-1})^{-1} = e
\end{equation}

\qed
\end{document}